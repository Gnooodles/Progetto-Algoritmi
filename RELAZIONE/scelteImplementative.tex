\section{Scelte Implementative}
Di seguito esponiamo le principali scelte implementative adottate nella realizzazione del codice per la misurazione empirica dei tempi di esecuzione, riportando brevemente an- che i principi di funzionamento alla base degli algoritmi di selezione. Inoltre, verranno sottolineate le differenze che caratterizzano le varianti di uno stesso processo di selezione.

\subsection{Aspetti generali}
?????

\subsection{Generatore test}
??????

\subsection{Clock e tempo minimo di esecuzione}
Prima di poter procedere al calcolo vero e proprio dei tempi di esecuzione, abbiamo dovuto 
ricavare la risoluzione del clock di sistema, 
indispensabile per il controllo degli errori
di misura.
Pertanto, è stata usata la funzione \textit{\itshape clock\_gettime}  della libreria time.h che,
grazie all’opzione \textsc{\scshape clock\_monotonic} , permette l’utilizzo di un orologio di tipo monotono.
Ottenuta dunque la risoluzione, ricavata dalla mediana di un buon numero di misurazioni,
è possibile calcolare un tempo minimo di esecuzione che assicuri l’accuratezza richiesta.
In particolare, si procede ad applicare la seguente formula 
$T_{min} \leq res\times(1+\frac{1}{\epsilon})$
così da garantire la massimizzazione dell’errore relativo . È importante specificare che l’errore usato per i 
calcoli corrisponde in realtà alla metà dell’errore relativo richiesto. Ciò sarà necessario per rispettare i 
limiti indicati nella consegna quando andremo ad effettuare i calcoli finali.

\subsection{Rilevazione tempi di esecuzione}
All’interno dei due cicli for innestati della procedura principale, viene rilevato il 
tempo di esecuzione di uno degli algoritmi su un vettore di dimensione e seme di generazione 
prestabiliti. La misurazione vera e propria è effettuata dalla funzione \textit{\itshape getExecutionTimes}.
Questa, in base agli argomenti ricevuti, provvede a ripetere il processo di misurazione per il 
numero di volte richiesto, restituendo alla fine la serie dei tempi registrarti. 
Ogni iterazione, in seguito alla corretta generazione di un array di input, consiste nella copiatura 
di quest’ultimo e nell’esecuzione dell’algoritmo di selezione, ripetendo eventualmente questi due passaggi 
fino al raggiungimento del tempo minimo di esecuzione precedentemente discusso o al superamento del tempo 
limite imposto dall’utente. Questo è un’indicazione del tempo massimo che l’utente è disposto ad aspettare 
per l’intero processo di misurazione sullo specifico input e ha lo scopo di evitare lunghe attese nel caso di 
algoritmi inefficienti o errati. In particolare, nel caso in cui questa limitazione venga superata, la procedura 
restituirà il valore \textsc{\scshape null} .

\subsection{Tempo generazione vettore}

\subsection{Calcoli finali ed outputs}
