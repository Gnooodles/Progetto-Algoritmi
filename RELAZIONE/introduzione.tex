\section{Introduzione}
Per la realizzazione del progetto del corso di “Algoritmi e Strutture Dati”, 
abbiamo trattato lo studio, l'implementazione e 
analisi empirica del tempo medio di esecuzione di tre algoritmi di selezione: 

“quick select”, “heap select” e “median-of-medians select”. 
Questi sono stati, dunque, adeguatamente modificati per rispettare i requisiti della consegna:

dato in input un array di numeri interi, non ordinato, e un indice k, restituire in output l'elemento in posizione k se l'array venisse ordinato.
Sono quindi state effettuate diverse misurazioni su input generati in maniera pseudocasuale con le due seguenti modalità:
???????dimensione del vettore variabile e indice fisso, e dimensione array fissata con indice variabile, concentrandoci in particolare su tre casi di studio che verranno descritti e analizzati in seguito.




